\documentclass[12pt]{article}
%% arXiv paper template by Flip Tanedo
%% last updated: Dec 2016


%%%%%%%%%%%%%%%%%%%%%%%%%%%%%
%%%  THE USUAL PACKAGES  %%%%
%%%%%%%%%%%%%%%%%%%%%%%%%%%%%

\usepackage{amsmath}
\usepackage{amssymb}
\usepackage{amsfonts}
\usepackage{graphicx}
\usepackage{xcolor}
\usepackage{nopageno}
\usepackage{enumerate}
\usepackage{parskip}

%%%%%%%%%%%%%%%%%%%%%%%%%%%%%%%%%
%%%  UNUSUAL PACKAGES        %%%%
%%%  Uncomment as necessary. %%%%
%%%%%%%%%%%%%%%%%%%%%%%%%%%%%%%%%

%% MATH AND PHYSICS SYMBOLS
%% ------------------------
%\usepackage{slashed}       % \slashed{k}
%\usepackage{mathrsfs}      % Weinberg-esque letters
%\usepackage{youngtab}	    % Young Tableaux
%\usepackage{pifont}        % check marks
\usepackage{bbm}           % \mathbbm{1} incomp. w/ XeLaTeX
%\usepackage[normalem]{ulem} % for \sout


%% CONTENT FORMAT AND DESIGN (below for general formatting)
%% --------------------------------------------------------
\usepackage{lipsum}        % block of text (formatting test)
%\usepackage{color}         % \color{...}, colored text
%\usepackage{framed}        % boxed remarks
%\usepackage{subcaption}    % subfigures; subfig depreciated
%\usepackage{paralist}      % compactitem
%\usepackage{appendix}      % subappendices
%\usepackage{cite}          % group cites (conflict: collref)
%\usepackage{tocloft}       % Table of Contents

%% TABLES IN LaTeX
%% ---------------
%\usepackage{booktabs}      % professional tables
%\usepackage{nicefrac}      % fractions in tables,
%\usepackage{multirow}      % multirow elements in a table
%\usepackage{arydshln} 	    % dashed lines in arrays

%% Other Packages and Notes
%% ------------------------
%\usepackage[font=small]{caption} % caption font is small



\renewcommand{\thesection}{}
\renewcommand{\thesubsection}{\arabic{subsection}}

%%%%%%%%%%%%%%%%%%%%%%%%%%%%%%%%%%%%%%%%%%%%%%%
%%%  PAGE FORMATTING and (RE)NEW COMMANDS  %%%%
%%%%%%%%%%%%%%%%%%%%%%%%%%%%%%%%%%%%%%%%%%%%%%%

\usepackage[margin=2cm]{geometry}   % reasonable margins

\graphicspath{{figures/}}	        % set directory for figures

% for capitalized things
\newcommand{\acro}[1]{\textsc{\MakeLowercase{#1}}}

\numberwithin{equation}{subsection}    % set equation numbering
\renewcommand{\tilde}{\widetilde}   % tilde over characters
\renewcommand{\vec}[1]{\mathbf{#1}} % vectors are boldface

\newcommand{\dbar}{d\mkern-6mu\mathchar'26}    % for d/2pi
\newcommand{\ket}[1]{\left|#1\right\rangle}    % <#1|
\newcommand{\bra}[1]{\left\langle#1\right|}    % |#1>
\newcommand{\Xmark}{\text{\sffamily X}}        % cross out


\let\olditemize\itemize
\renewcommand{\itemize}{
  \olditemize
  \setlength{\itemsep}{1pt}
  \setlength{\parskip}{0pt}
  \setlength{\parsep}{0pt}
}


% Commands for temporary comments
\newcommand{\comment}[2]{\textcolor{red}{[\textbf{#1} #2]}}
\newcommand{\flip}[1]{{\color{red} [\textbf{Flip}: {#1}]}}
\newcommand{\email}[1]{\texttt{\href{mailto:#1}{#1}}}

\newenvironment{institutions}[1][2em]{\begin{list}{}{\setlength\leftmargin{#1}\setlength\rightmargin{#1}}\item[]}{\end{list}}


\usepackage{fancyhdr}		% to put preprint number



%%%%%%%%%%%%%%%%%%%
%%%  HYPERREF  %%%%
%%%%%%%%%%%%%%%%%%%

%% This package has to be at the end; can lead to conflicts
\usepackage{microtype}
\usepackage[
	colorlinks=true,
	citecolor=black,
	linkcolor=black,
	urlcolor=green!50!black,
	hypertexnames=false]{hyperref}



%%%%%%%%%%%%%%%%%%%%%
%%%  TITLE DATA  %%%%
%%%%%%%%%%%%%%%%%%%%%

\begin{document}


\begin{center}

    {\Large \textsc{Homework 1:}
    \textbf{Representations of SU(2) and SU(3)}}

\end{center}

\vskip .4cm

\noindent
\begin{tabular*}{\textwidth}{rl}
	\textsc{Course:}& Physics 262, \emph{Group Theory for Physicists} (Fall 2019)
	\\
	\textsc{Instructor:}& Professor Flip Tanedo (\email{flip.tanedo@ucr.edu})
	\\
	\textsc{Due by:}& Be ready to discuss on Friday, Feb 8
	\\
	\textsc{Updated:}& Jan 24, 11:30am

	%
\end{tabular*}



\subsection{The spin-1 representation of SU(2)}
%

The spin-1 representation of SU(2) is the three dimensional representation with highest weight $j=1$ and states $\ket{1}$, $\ket{0}$, and $\ket{-1}$. A vector in this space is:
\begin{align}
  \begin{pmatrix}
    a \\ b \\ c
  \end{pmatrix}
  &=
  a \ket{1} + b \ket{0} + c\ket{-1} \ .
\end{align}

\subsubsection{Normalizations}

We chose a normalization of our states so that:
\begin{align}
  T^- \ket{m} & = N_m \ket{m-1} \\
  T^+ \ket{m-1} & = N_m \ket{m} \ ,
\end{align}
where $\ket{n}$ is orthonormal. Write out $N_m$ for each of the weights $m=\pm 1, 0$.


\subsubsection{Raising, lowering, Cartan}

Write out the explicit form of the generators in this representation: $d(T^+)$, $d(T^-)$, $d(T^3)$.

\subsubsection{Generators in the `usual' basis}

Write out the explicit form of the generators in this representation: $d(T^1)$, $d(T^2)$, $d(T^3)$. Recall that:
\begin{align}
  d(T^\pm) &= d(T^1) \pm i d(T^2) \ .
\end{align}

\subsection{The adjoint representation of SU(2)}

\flip{Updated 1/24: typo on $\varepsilon^{123}$, $(-i)$ on $\text{ad}(T^a)^{bc} = -if^{abc}$. Version below is corrected.}

The \textbf{adjoint} representation is one where the generators themselves are states. The action of a generator (as a matrix) on a generator (as a state) is given by the \textbf{structure constant}, $f^{abc}$. Recall that the structure constant is defined by
\begin{align}
  [T^a, T^b] &= i f^{abc}T^c \ .
\end{align}
For SU(2) $f^{abc} = \varepsilon^{abc}$, the totally antisymmetric tensor with $\varepsilon^{123} = 1$. The matrices of the adjoint representation are:
\begin{align}
  \text{ad}(T^a)^{bc} = -i f^{abc} \ .
\end{align}
In other words, the action of a generator in the adjoint representation $\text{ad}(T^a)$ on a state $\ket{T^b}$ is
\begin{align}
  \text{ad}(T^a)\ket{T^b} = -if^{abc}\ket{T^c} \ .
   % = -\ket{[T^a,T^b]}
\end{align}
\flip{Update: removed irrelevant information.}

\subsubsection{Dimension of the adjoint representation}

What is the dimension of the adjoint representation? (How many basis states are there?) \textsc{Answer}: three. (You may want to write one sentence explaining why the answer is three. If the answer is more than one sentence, then it's probably wrong.)


\subsubsection{Dimension of the adjoint representation}

\flip{Update: clarified the basis. Why is the $\ket{T^\pm}, \ket{T^3}$ basis a useful one to use instead of $\ket{T^{1,2}}, \ket{T^3}$? (Think about the weight of the state.)}.

Write out the explicit matrix form of $\text{ad}(T^3)$ acting on a basis of states $\ket{T^\pm}, \ket{T^3}$. Confirm that it matches $d(T^3)$ in the spin-1 representation acting on states $\ket{m=\pm1}, \ket{m=0}$. Write out the explicit matrix forms of $\text{ad}(T^1)$ and $\text{ad}(T^2)$ acting on the basis $\ket{T^\pm}, \ket{T^3}$. Observe that this is \emph{not} quite the same as $d(T^1)$ and $d(T^2)$ of the spin-1 repesentation.

\subsubsection{What gives?}

Confirm that $\text{ad}(T^1)$, $\text{ad}(T^2)$, and $\text{ad}(T^3)$ satisfy the SU(2) commutation relations:
\begin{align}
  [\text{ad}(T^a), \text{ad}(T^b)] = i\varepsilon^{abc}\text{ad}(T^c) \ .
\end{align}
Confirm that if some set of matrices $d(T)$ are a representation---that is, they satisfy the algebra's commutation relations---then another set of matrices that differ by a unitary transformation, $\tilde d(T) \equiv U d(T) U^\dag$, is also a representation. Write down the matrix $U$ that transforms the spin-1 representation into the adjoint representation. This proves that the spin-1 and adjoint representations are, in fact, the same. The difference between them is purely ``cosmetic.'' Hint: \url{https://physics.stackexchange.com/q/279880/166736}. The difference has to do with the spherical basis of rotations.

\flip{Remark: the basis $T^\pm$, $T^3$ forms an algebra that has different structure constants than $\varepsilon^{abc}$. That's okay. This basis is useful for building the representations of the group and understanding how they transform, but $T^\pm$ is not a Hermitian matrix.}

% \subsection{SU(3) with different Cartan bases}
%
% \subsubsection{A weird basis}
%
% Suppose we chose a weird basis for the Cartan subalgebra of SU(3):
% \begin{align}
%   H_1 &= \frac{1}{2}\text{diag}(0, 1, -1)
%   &
%   H_2 &= \frac{1}{2\sqrt{3}}\text{diag}(-2, 1, 1) \ .
% \end{align}
% These generators have ```quantum numbers'' $(r,s)$:
% \begin{align}
%   H_1 \ket{r,s} &= r \ket{r,s} &
%   H_2 \ket{r,s} &= s \ket{r,s}  \ .
% \end{align}
% For each set of raising and lowering operators, $X^\pm$, work out $[X^+, X^-]$, $[H_1,X^\pm]$, and $[H_2,X^\pm]$. Using what we know about SU(2), identify the conditions on $(r,s)_\text{max}$, the ``highest weight'' of the representation. Draw the weight diagram for the \textbf{fundamental representation} in $(r,s)$ weight space. How is this relsssssated to the $(p,q)$ weight space we introduced in Lecture 4?
%
% \subsubsection{A stupid basis}
%
% Here's a stupid basis for the Cartan subalgebra of SU(3). It's close to the basis we used in class, but improperly normalized.
% \begin{align}
%   H_1 &= \frac{1}{2}\text{diag}(1, -1,)
%   &
%   H_2 &= \frac{1}{2}\text{diag}(1, 1, -2) \ .
% \end{align}
% What are the consequences on the weight diagram? Draw the \emph{correct} weight diagram for the fundamental representation (what we did in class) and overlay it with the \emph{stupid} weight diagram from this improperly normalized set of Cartan generators.
%
% \subsubsection{Another stupid basis}
%
% Here's another stupid basis for the Cartan subalgebra of SU(3).
% \begin{align}
%   H_1 &= \frac{1}{2}\text{diag}(1,-1, 0)
%   &
%   H_2 &= \frac{1}{2\sqrt{3}}\text{diag}(-2, 1, 1) \ .
% \end{align}
% What's goes wrong with this basis? \textsc{Hint}: notice that these generators are not `orthogonal' with respect to the Killing form, $\kappa(T^a, T^b) = \text{Tr}(T^aT^b)$. That is: $\kappa(H_1, H_2) \neq 0$. What implications does this have when drawing the weight diagram?


\subsection{Properties of Algebras}

Show that:
\begin{itemize}
  \item If a matrix Lie group is defined to be \textbf{special} (unit determinant), then the algebra is made up of traceless matrices.
  \item If a matrix Lie group is defined to be \textbf{unitary}, then the algebra is made up of Hermitian matrices.
\end{itemize}

\subsection{Adjoint representation of a group, algebra}

\flip{Update: Clarification and discussion.} Let $G$ be a matrix Lie group and call the generators $T^a$. We can write elements of $G$ that are ``sufficiently close to the origin'' as $g = \exp(-i\theta^a T^a)$. The adjoint representation of the \emph{group} is the action of $G$ on elements of its algebra:
\begin{align}
  \text{Ad}(g) \ket{X} = g X g^{-1} \ ,
\end{align}
where $X = x^a T^a$ is an element of the algebra.
Show that if $g = \exp(-i\epsilon^a T^a)$ for small angles $\epsilon^a$, the adjoint action of the \emph{group} reduces to the adjoint action of the \emph{algebra}:
\begin{align}
  Ad(g) \ket{X} = \left(1 + c^a \text{ad}(T^a) + \mathcal O(\epsilon^2) \right) X \ .
\end{align}
Determine what $c^a$ is. Observe that the action of $\text{ad}(T^a)$ on $X$ is a commutator:
\begin{align}
    \text{ad}(T^a)\ket{T^b} = -if^{acb}\ket{T^c} = \ket{[T^a,T^b]} \ .
    \label{eq:commutator}
\end{align}
\flip{Correction: 1/30: minus sign on (\ref{eq:commutator}), see Georgi (6.8)}
%
\textsc{Hint}: See Gutowski sections 2.14 - 2.26.

\section*{Extra Credit}

These problems are for your own edification. You are encouraged to explore them according to your own personal and research interests. \textbf{Relevant}: \url{https://youtu.be/0obMRztklqU}.


\subsection*{BCH}

Derive the Baker-Campbell-Hausdorff formula. Start by looking up what the Baker-Campbell-Hausdorff formula is. I don't have anything deep to say about this, but going through the derivation once gives you a feel for how to think of the group and algebra as a manifold and tangent space.


\end{document}
