%% LaTeX Paper Template, Flip Tanedo (flip.tanedo@ucr.edu)
%% last updated: September 2018

\documentclass[12pt]{article}

%%%%%%%%%%%%%%%%%%%%%%%%%%
%%%  COMMON PACKAGES  %%%%
%%%%%%%%%%%%%%%%%%%%%%%%%%

\usepackage{amsmath}
\usepackage{amssymb}
\usepackage{amsfonts}
\usepackage{graphicx}
\usepackage[utf8]{inputenc}	
%\usepackage{amsthm}				

%%%%%%%%%%%%%%%%%%%%%%%%%%%%%%%%%
%%%  UNUSUAL PACKAGES        %%%%
%%%  Uncomment as necessary. %%%%
%%%%%%%%%%%%%%%%%%%%%%%%%%%%%%%%%

%% MATH AND PHYSICS SYMBOLS
%% ------------------------
%\usepackage{slashed}       % \slashed{k}
%\usepackage{mathrsfs}      % Weinberg-esque
%\usepackage{youngtab}	    % Young Tableaux
%\usepackage{pifont}        % check marks
%\usepackage{bbm}           % chalkboard bold
%\usepackage[normalem]{ulem} % for \sout
%\usepackage{cancel}

%% CONTENT FORMAT AND DESIGN
%% -------------------------
\usepackage[dvipsnames]{xcolor}
\usepackage{fancyhdr}		% preprint number
\usepackage{lipsum}         % block of text
\usepackage{framed}			% boxed remarks
%\usepackage{subcaption}    % subfig depreciated
%\usepackage{paralist}      % compactitem
%\usepackage{appendix}      % subappendices
%\usepackage{cite}          % group cites 
%\usepackage{tocloft}       % Table of Contents	
%\usepackage{xspace}		% macro spacing
%\usepackage{adjustbox}		% For images in equations tex.stackexchange/270830


%\usepackage{listings}      % for code, eg.
%
% \begin{lstlisting} 
%	\lstset{
%		basicstyle=\ttfamily\footnotesize,
%		breaklines=true,
%		backgroundcolor=\color{gray!15!white}}


%% TABLES IN LaTeX
%% ---------------
\usepackage{booktabs}		% professional
\usepackage{nicefrac}		% fractions
%\usepackage{multirow}     
%\usepackage{arydshln}		% dashed lines


%% Other Packages and Notes
%% ------------------------
\usepackage[font=small]{caption} 
\usepackage{float}         % strict [H]


%% CUSTOM PACKAGES
%% ---------------
%\usepackage{tikzfeynman}   % Flip's rules Feynman Diagrams



%%%%%%%%%%%%%%%%%%%%%%%%%%%%%%
%%%  DOCUMENT PROPERTIES  %%%%
%%%%%%%%%%%%%%%%%%%%%%%%%%%%%%

\usepackage[margin=2cm]{geometry}   
\graphicspath{{figures/}}			
\numberwithin{equation}{section}    

%% References in two columns, smaller
\usepackage{multicol}
\usepackage{etoolbox}
\usepackage{relsize}
\patchcmd{\thebibliography}
  {\list}
  {\begin{multicols}{2}\smaller\list}
  {}
  {}
\appto{\endthebibliography}{\end{multicols}}
%% Alternative (one column, modify spacing):
%% https://wiki.math.cmu.edu/iki/wiki/tips/20140712-bibtex-spacing.html

% Change list spacing
% from: http://en.wikibooks.org/wiki/LaTeX/List_Structures#Line_spacing
\let\oldenumerate\enumerate
\renewcommand{\enumerate}{
  \oldenumerate
  \setlength{\itemsep}{1pt}
  \setlength{\parskip}{0pt}
  \setlength{\parsep}{0pt}
}

\let\olditemize\itemize
\renewcommand{\itemize}{
  \olditemize
  \setlength{\itemsep}{1pt}
  \setlength{\parskip}{0pt}
  \setlength{\parsep}{0pt}
}


%%%%%%%%%%%%%%%%%%%%%%%%%%%
%%%  (RE)NEW COMMANDS  %%%%
%%%%%%%%%%%%%%%%%%%%%%%%%%%

%% FOR `NOT SHOUTING' CAPS (e.g. acronyms)
%% ---------------------------------------
\newcommand{\acro}[1]{\textsc{\MakeLowercase{#1}}}    

%% COMMON PHYSICS MACROS
%% ---------------------
\renewcommand{\tilde}{\widetilde}   % tilde over characters
\renewcommand{\vec}[1]{\mathbf{#1}} % vectors are boldface
\newcommand{\dbar}{d\mkern-6mu\mathchar'26}    % for d/2pi
\newcommand{\ket}[1]{\left|#1\right\rangle}    % <#1|
\newcommand{\bra}[1]{\left\langle#1\right|}    % |#1>
\newcommand{\Xmark}{\text{\sffamily X}}        % cross out

%% COMMANDS FOR TEMPORARY COMMENTS
%% -------------------------------
\newcommand{\comment}[2]{\textcolor{red}{[\textbf{#1} #2]}}
\newcommand{\flip}[1]{{
	\color{green!50!black} \footnotesize [\textbf{\textsf{Flip}}: \textsf{#1}]
	}}


%% COMMANDS FOR TOP-MATTER
%% -----------------------
\newcommand{\email}[1]{\href{mailto:#1}{#1}}
\newenvironment{institutions}[1][2em]{\begin{list}{}{\setlength\leftmargin{#1}\setlength\rightmargin{#1}}\item[]}{\end{list}}


%% COMMANDS FOR LATEXDIFF
%% ----------------------
%% see http://bit.ly/1M74uwc
\providecommand{\DIFadd}[1]{{\protect\color{blue}#1}} %DIF PREAMBLE
\providecommand{\DIFdel}[1]{{\protect\color{red}\protect\scriptsize{#1}}}

%% REMARK: use latexdiff option --allow-spaces
%% for \frac, ref: http://bit.ly/1iFlujR


%%%%%%%%%%%%%%%%%%%%%%%%%%%%%%%%%%%%%%%%%%%%%%
%%%  TIKZ COMMANDS FOR EXTERNAL DIAGRAMS  %%%%
%%%  requires -shell-escape               %%%%
%%%  in texpad 1.7: prefs > shell esc sec %%%%
%%%%%%%%%%%%%%%%%%%%%%%%%%%%%%%%%%%%%%%%%%%%%%

%% For exporting tikz figures as into a ./tikz/ subfolder.
%% It is useful if you want pdf versions of the tikz diagrams or
%% if you need to speed up compilation of a large document with
%% many tikz diagrams.

%\write18{} % Careful with this!
%\usetikzlibrary{external}
%\tikzexternalize[prefix=tikz/] % folder for external pdfs


%%%%%%%%%%%%%%%%%%%
%%%  HYPERREF  %%%%
%%%%%%%%%%%%%%%%%%%

%% This package has to be at the end; can lead to conflicts

\usepackage[
	colorlinks=true,
	citecolor=green!50!black,
	linkcolor=NavyBlue!75!black,
	urlcolor=green!50!black,
	hypertexnames=false]{hyperref}


%%%%%%%%%%%%%%%%%%%%%
%%%  TITLE DATA  %%%%
%%%%%%%%%%%%%%%%%%%%%

%% PREPRINT NUMBER USING fancyhdr
%% Must set \thispagestyle{firststyle}
%% ----------------------------------------------
\renewcommand{\headrulewidth}{0pt} 	% no separator
\setlength{\headheight}{15pt} 		% min to avoid fancyhdr warning
\fancypagestyle{firststyle}{
	\rhead{\footnotesize%
	\texttt{UCR-FLIP-2019-PHYS262-01}%\\
%	\texttt{UCI-TR-2019-XX}
	}}

%% TOC overwrites fancyhdr, here's a fix
%% http://tex.stackexchange.com/questions/167828/difficult-with-fancyhdr-and-table-of-contents
\usepackage{etoc}
\renewcommand{\etocaftertitlehook}{\pagestyle{plain}}
\renewcommand{\etocaftertochook}{\thispagestyle{firststyle}}



\begin{document}

%\thispagestyle{empty}		% default if no preprint #
%\thispagestyle{firststyle} % to include preprint
							% overwritten by etoc

\begin{center}

    {\huge \bf Physics 262: SU(2) Conventions}

    \vskip .7cm

%% SINGLE AUTHOR FORMAT
%% --------------------
	\textbf{Flip Tanedo} \\
	\texttt{\footnotesize \email{flip.tanedo@ucr.edu}}

	\vspace{-.8em}
    \begin{institutions}[2.25cm]
    \footnotesize
    {\it 
	    Department of Physics \& Astronomy, 
	    University of  California, Riverside, 
	    {CA} 92521	    
	    }    
    \end{institutions}


\end{center}




%%%%%%%%%%%%%%%%%%%%%
%%%  ABSTRACT    %%%%
%%%%%%%%%%%%%%%%%%%%%

\begin{abstract}
\noindent 
Some conventions for group theory. Corrections and additions welcome. We roughly follow chapter 3.2 of Georgi with slightly different conventions.
\end{abstract}



\small
\setcounter{tocdepth}{2}
\tableofcontents
\normalsize
%\clearpage

\section{Basis of Generators}

The SU(2) algebra is
\begin{align}
	[T^a, T^b] = i \varepsilon^{abc} T^c \ .
\end{align}
Let us decompose these generators into a pair of raising/lowering operators $T^\pm$ and the diagonal generator, $T^3$. We choose to define the raising/lowering operators as:
\begin{align}
	T^\pm = T^1 \pm i T^2 \,
\end{align}
note that Georgi's definition differs by a factor of $1/\sqrt{2}$.  These satisfy the commutation relations:
\begin{align}
	[T^3, T^\pm] & = \pm T^\pm 
	\\
	[T^+, T-] & = 2T^3 \ .
\end{align}

\section{Weights}

The \textbf{weight}, $m$, of a state is the $T^3$ eigenvalue:
\begin{align}
	T^3\ket{m} = m \ket{m} \ .
\end{align}
The weight is an index for the different components of the representation. 

\section{Raising/Lowering Weights}

The $T^\pm$ operators increment the weight of a state by $\pm 1$. We can see this by applying $T^3$ to the state $T^\pm\ket{m}$:
\begin{align}
	T^3 T^\pm \ket{m} &= \left([T^3, T^\pm] + T^\pm T^3 \right) \ket{m} 
	= \left( m \pm 1 \right) T^\pm \ket{m} \ .
\end{align}
The result is that acting on a state with $T^\pm$ moves you to a new 

\section{Finite Dimensional Representations}

Let us examine representations of SU(2) that have a finite number of weights---that is, a finite number of components. The $T^\pm$ operators mean that given a state with some state $\ket{m}$, you can ``rotate'' to a different state $\ket{m\pm 1}$. We are interested in finite dimensional representations---objects with only a finite number of components (weights) that rotate into each other. Thus this ladder must truncate for some maximum and minimum weights. 
%
Let $j$ be the \textbf{maximum weight} of the representation:
\begin{align}
	T^+\ket{j} &= 0 \ .
\end{align}
Let us sequentially derive the states of the representation by applying the lowering operator. What we want to find is the value of the \emph{minimum weight} so that we will know that we have found all of the states.

We can assume that $\ket{j}$ is normalized. However, we have no guarantee that $T^-\ket{j}$ is normalized. Thus let us define a normalization
\begin{align}
	T^- \ket{j} &= N_j \ket{j-1} \ ,
\end{align}
where $\ket{j-1}$ is normalized. Let's look at this first state below the maximum weight state.

\subsection{Normalization of $\ket{j-1}$}

We determine the normalization $N_j$ by calculating the norm $|| T^-\ket{j}||$:
\begin{align}
	\bra{j} \left(T^-\right)^\dag T^- \ket{j} 
	& = \bra{j} T^+ T^- \ket{j} \\
	& = \bra{j} \left( [T^+, T^-] + T^-T^+ \right) \ket{j} \\
	& = \bra{j} 2T^3 \ket{j} \\
	& = 2j \ .
\end{align}
We have used $T^+\ket{j} = 0$. Using the definition of the normalized state $|j-1\rangle$, this must equal:
\begin{align}
	2j & = |N_j|^2 \langle j -1 | j+1 \rangle \ ,
\end{align}
so that we have found the normalization
\begin{align}
	N_ j = \sqrt{2j} \ .
\end{align}

Next we should check the normalization of the $T^+\ket{j-1}$. To do this, we use the one strategy available: \emph{use the commutation relations to simplify the expression}. Use the facts that
\begin{enumerate}
	\item $T^+$ annihilates $\ket{j}$
	\item $T^3\ket{m} = m$.
\end{enumerate}
Thus the raising of $\ket{j-1}$ gives:
\begin{align}
	T^+ \ket{j-1} &= T^+ \left( \frac{1}{N_j} T^- \ket{j} \right) \\
	& = \frac{1}{N_j}\left( [T^+,T^-] + T^-T^+ \right) \ket{j} \\
	& = \frac{2j}{N_j} \ket{j} \\
	& = N_j \ket{j} \ .
\end{align}
In the last line we explicitly use $N_j = \sqrt{2j}$. We observe that raising $\ket{j-1}$ and lowering $\ket{j}$ produce the same normalization:
\begin{align}
	T^- \ket{j} &= N_j \ket{j-1}\\
	T^+ \ket{j-1} &= N_j \ket{j} \ .
\end{align}

\subsection{Recursion relation}

Now suppose that for some general weight $\ket{m}$ in our representation, we know that:
\begin{align}
	T^- \ket{m+1} &= N_{m+1} \ket{m} \\
	T^+ \ket{m} &= N_{m+1} \ket{m+1} \ ,
\end{align}
where $N_{m+1}$ is known and the states $\ket{m}$, $\ket{m+1}$ are normalized. This is the case with the `top of the ladder,' $(m+1) = 1$. We now derive $N_m$ and show that the same relation holds for $m\to (m-1)$. $N_m$ is defined by
\begin{align}
	T^- \ket{m} = N_m \ket{m-1} \ .
\end{align}
$N_m$ is fixed by requiring that $\ket{m-1}$ is normalized, $\langle m-1 | m-1\rangle = 1$. Then we have:
\begin{align}
	N_m^2  &= \bra{m} T^+ T^- \ket{m} \\
	&= \bra{m} \left( [T^+,T^-] + T^-T^+ \right) \ket{m}\\
	&= 2m +  N_{m+1}^2 \ . \label{eq:Nm:recursion}
\end{align}
We thus have a recursion relation for $N_m$ in terms of $N_{m+1}$.

To complete the recursion, we must prove that
\begin{align}
	T^+ \ket{m-1} & = N_m \ket{m} \ .
\end{align}
The left-hand side is
\begin{align}
	T^+ \ket{m-1} &= T^+ \left(\frac{1}{N_m} T^- \ket{m}\right)\\
	& = \frac{1}{N_m} \left( [T+,T-] + T^-T^+ \right) \ket{m} \\
	& = \frac{1}{N_m} \left( 2m \ket{m} + N_{m+1}T^- \ket{m+1} \right) \\
	& = \frac{1}{N_m} \left( 2m \ket{m} + N_{m+1}^2 \ket{m} \right) \\
	& = N_m \ket{m} \ ,
\end{align}
where we have used the explicit result, (\ref{eq:Nm:recursion}). 

\subsection{Closed expression for the normalization}

We can imagine writing down the relations for the squared normalizations:
\begin{align}
	\begin{array}{ccccc}
	N_j^2 & & &=& 2j \\	
	N_{j-1}^2 & -  & N_j^2 &=& 2(j-1) \\
	N_{j-2}^2 & -  & N_{j-1}^2 &=& 2(j-2) \\
	\vdots & & & & \vdots \\
	N_{j-k}^2 & -  & N_{j-k-1}^2 &=& 2(j-k) \\
	\end{array}
\end{align}
Sum both sides of the equations and recognizing that there are $(k+1)$ terms,
\begin{align}
	N_{j-k}^2 &= 2(k-1)j -  2\frac{k(k+1)}2\\
	& = (k+1)(2j - k) \ .
\end{align}
Rather than using ``$k$ steps below the highest weight, '' it is more tidy to use the normalization of the $m^\text{th}$ rung of the ladder via $k = j-m$:
\begin{align}
	N_m = \sqrt{(j-m+1)(j+m)} \ .
\end{align}

\section{Bottom of the ladder}

In order for the representation to be finite, we need the ladder to truncate at the bottom. There is thus some a limit to the number of times you can hit the highest weight state $\ket{j}$ with $T^-$ before it annihilates. Let this number be $\ell$. Then we want to use
\begin{align}
	T^- \ket{j-\ell} &= 0
\end{align}
with the normalization $N_{j-\ell}$ to determine what this $\ell$ is. 
\begin{align}
	N_{j-\ell} = \sqrt{(\ell+1)(2j - \ell)} \ .
\end{align}
Observe that $(\ell+1) >0$ since $\ell$ is a positive natural number. Then the requirement that the ladder terminates, $N_{j-\ell} = 0$, requires 
\begin{align}
	\ell = 2j \ .
\end{align}
From this we observe:
\begin{enumerate}
	\item The highest weight $m_\text{max} = j$ is half integer. It can be written as $\ell/2$ for some positive integer $\ell$. 
	\item The lowest weight is simply $m_\text{min} = -j$. 
\end{enumerate}




%% Appendices
% \appendix


%% Bibliography
\bibliographystyle{utcaps} 	% arXiv hyperlinks
% \bibliography{bib title without .bib}


\end{document}